
\usepackage[spanish,activeacute,es-tabla]{babel}   
\usepackage[utf8]{inputenc}                        
\usepackage{ifthen}                                
\usepackage{listings}                              
\usepackage{dsfont}                                
\usepackage{subcaption}                            
\usepackage{amsmath}                               
\usepackage[strict]{changepage}                    
\usepackage[top=1cm,bottom=2cm,left=1cm,right=1cm]{geometry}
\usepackage{color}                                 
\usepackage{graphicx}                              
\usepackage{relsize}                               
\usepackage{exscale}                               
\usepackage[dvipsnames]{xcolor}                    

% ==================================================================
\newenvironment{IndentedBlock}{
  \begin{list}{}{
    \leftmargin=2em   % sangrado izquierdo
    \rightmargin=0em
    \topsep=0pt
    \partopsep=0pt
    \parsep=3pt
    \itemsep=0pt
  }\item\relax
}{
  \end{list}
}

% ==================================================================

\newcommand{\In}[2]{\textsf{in} #1 : #2}       % in nombre : tipo
\newcommand{\Out}[2]{\textsf{out} #1 : #2}     % out nombre : tipo
\newcommand{\Inout}[2]{\textsf{inout} #1 : #2} % inout nombre : tipo

% ==================================================================
% redefine requiere para mantener indentación y corchetes
\newcommand{\requiere}[2][]{%
  \noindent\textbf{\ttfamily requiere}%
  \ifthenelse{\equal{#1}{}}{}{ {\normalfont\ttfamily #1}}%
  \ \{% abre corchete
    \begin{IndentedBlock}
      \(\begin{aligned}#2\end{aligned}\)
    \end{IndentedBlock}
  \}\par\medskip
}

\newcommand{\asegura}[2][]{%
  \noindent\textbf{\ttfamily asegura}%
  \ifthenelse{\equal{#1}{}}{}{ {\normalfont\ttfamily #1}}%
  \ \{% abre corchete
    \begin{IndentedBlock}
      \(\begin{aligned}#2\end{aligned}\)
    \end{IndentedBlock}
  \}\par\medskip
}


% ==================================================================
% 5. Entornos para TAD, predicado, auxiliar y procedimiento
% ------------------------------------------------------------------

% --- Predicado ---
\newcommand{\predheader}[2]{%
  \noindent\textbf{\ttfamily pred}\ {\ttfamily #1}(#2)%
}
\newenvironment{pred}[2]{%
  \predheader{#1}{#2}\{%
    \begin{IndentedBlock}
  }{%
    \end{IndentedBlock}
  \}\par
}

% --- Auxiliar ---
\newcommand{\auxheader}[3]{%
  \noindent\textbf{\ttfamily aux}\ {\ttfamily #1}(#2)\ :\ {\ttfamily #3}\ =%
}
\newenvironment{aux}[3]{%
  \auxheader{#1}{#2}{#3}%
  \begin{IndentedBlock}
}{%
  \end{IndentedBlock}\par
}

% --- Procedimiento ---
\newcommand{\procheader}[3]{%
  \noindent\textbf{\ttfamily proc}\ {\ttfamily #1}(#2)%
  \ifthenelse{\equal{#3}{}}{}{ : {\ttfamily #3}}%
}
\newenvironment{proc}[3]{%
  \procheader{#1}{#2}{#3}\{%
    \begin{IndentedBlock}
  }{%
    \end{IndentedBlock}
  \}\par
}

% --- TAD ---
\newcommand{\tadheader}[2]{%
  \noindent\textbf{\ttfamily TAD}\ {\ttfamily #1}%
  \ifthenelse{\equal{#2}{}}{}{%
    {$\langle#2\rangle$}%
  }%
}
\newenvironment{tad}[2]{%
  \tadheader{#1}{#2}\{%
    \begin{IndentedBlock}
  }{%
    \end{IndentedBlock}
  \}\par
}

% ==================================================================
\newcommand{\tnat}{\ensuremath{\mathds{N}}}
\newcommand{\tint}{\ensuremath{\mathds{Z}}}
\newcommand{\treal}{\ensuremath{\mathds{R}}}
\newcommand{\tbool}{\ensuremath{\mathsf{Bool}}}
\newcommand{\tfloat}{\ensuremath{\mathsf{Float}}}
\newcommand{\tchar}{\ensuremath{\mathsf{Char}}}
\newcommand{\tstring}{\ensuremath{\mathsf{String}}}

\newcommand{\ttuple}[1]{\ensuremath{\mathsf{Tuple}\langle#1\rangle}}
\newcommand{\tseq}[1]{\ensuremath{\mathsf{Seq}\langle#1\rangle}}
\newcommand{\tconj}[1]{\ensuremath{\mathsf{Conj}\langle#1\rangle}}
\newcommand{\tdict}[1]{\ensuremath{\mathsf{Dict}\langle#1\rangle}}
\newcommand{\tstruct}[1]{\ensuremath{\mathsf{Struct}\langle#1\rangle}}

% ==================================================================
\newcommand{\comment}[1]{\textcolor{Periwinkle}{// #1}\\}
\newcommand{\bigsum}[2]{\ensuremath{\sum\limits_{#1}^{#2}}}
\newcommand{\bigunion}[2]{\ensuremath{\bigcup\limits_{#1}^{#2}}}
\newcommand{\bbpar}[1]{\bigl(#1\bigr)}
\newcommand{\bpar}[1]{\Bigl(#1\Bigr)}

\newcommand{\True}{\ensuremath{\mathrm{true}}}
\newcommand{\False}{\ensuremath{\mathrm{false}}}
\newcommand{\Then}{\ensuremath{\rightarrow}}
\newcommand{\Iff}{\ensuremath{\leftrightarrow}}
\newcommand{\implica}{\ensuremath{\longrightarrow}}
\newcommand{\IfThenElse}[3]{\ensuremath{\mathsf{if}\ #1\ \mathsf{then}\ #2\ \mathsf{else}\ #3\ \mathsf{fi}}}
\newcommand{\yLuego}{\land _L}
\newcommand{\oLuego}{\lor _L}
\newcommand{\implicaLuego}{\implica _L}
\newcommand{\IfThenElseFi}[3]{%
  \ensuremath{\mathsf{if}\ #1\ \mathsf{then}\ #2\ \mathsf{else}\ #3\ \mathsf{fi}}%
}
\newcommand{\nsum}[1][1.4]{%
  \mathop{\raisebox{-#1\depthofsumsign}{\scalebox{#1}{$\displaystyle\sum$}}}%
}

% ==================================================================
\newcommand{\alias}[2]{\noindent #1 \textbf{ES}\ #2}  
\newcommand{\obs}[2]{%
  \par\noindent\textbf{obs}\ \ensuremath{#1 : #2}\par
}
\newcommand{\call}[2]{\ensuremath{\mathrm{#1}(#2)}}


% ==================================================================
\renewcommand{\lstlistingname}{Código} 
\lstset{
  language=Java,                    
  morekeywords={endif,endwhile,skip},
  basewidth={0.47em,0.40em},
  columns=fixed,flexiblecolumns=false,
  tabsize=4,breaklines=true,
  numbers=left,numberstyle=\tiny,
  stepnumber=1,numbersep=9pt,
  frame=l,framesep=3pt,captionpos=b,
}
